%\VignetteIndexEntry{xtable Gallery}
%\VignetteDepends{xtable}
%\VignetteKeywords{LaTeX,HTML,table}
%\VignettePackage{xtable}

%**************************************************************************
%
% # $Id:$

% $Revision:  $
% $Author: $
% $Date:  $


\documentclass[letterpaper]{article}

\title{
The xtable gallery
}
\author{Jonathan Swinton <jonathan@swintons.net>\\ with small contributions from others}

\usepackage{Sweave}
%removed a small part from SweaveOpts that prevented compilation

\usepackage{rotating}
\usepackage{longtable}
\usepackage{booktabs}
\usepackage{tabularx}
%\usepackage{hyperref}
\begin{document}
\input{xtableGallery-concordance}

\maketitle
\section{Summary}
This document gives a gallery of tables which can be made
by using the {\tt xtable} package to create \LaTeX\ output.
It doubles as a regression check for the package.

\begin{Schunk}
\begin{Sinput}
> library(xtable)
> options(xtable.timestamp="")
\end{Sinput}
\end{Schunk}

\section{Gallery}
\subsection{Data frame}
Load example dataset
\begin{Schunk}
\begin{Sinput}
> data(tli)
> ## Demonstrate data.frame
> tli.table <- xtable(tli[1:10,])
> digits(tli.table)[c(2,6)] <- 0
\end{Sinput}
\end{Schunk}
\begin{Schunk}
\begin{Sinput}
> print(tli.table,floating=FALSE)
\end{Sinput}
% latex table generated in R 3.1.1 by xtable 1.7-3 package
% 
\begin{tabular}{rrlllr}
  \hline
 & grade & sex & disadvg & ethnicty & tlimth \\ 
  \hline
1 & 6 & M & YES & HISPANIC & 43 \\ 
  2 & 7 & M & NO & BLACK & 88 \\ 
  3 & 5 & F & YES & HISPANIC & 34 \\ 
  4 & 3 & M & YES & HISPANIC & 65 \\ 
  5 & 8 & M & YES & WHITE & 75 \\ 
  6 & 5 & M & NO & BLACK & 74 \\ 
  7 & 8 & F & YES & HISPANIC & 72 \\ 
  8 & 4 & M & YES & BLACK & 79 \\ 
  9 & 6 & M & NO & WHITE & 88 \\ 
  10 & 7 & M & YES & HISPANIC & 87 \\ 
   \hline
\end{tabular}\end{Schunk}

\subsection{Matrix}
\begin{Schunk}
\begin{Sinput}
> design.matrix <- model.matrix(~ sex*grade, data=tli[1:10,])
> design.table <- xtable(design.matrix)
\end{Sinput}
\end{Schunk}
\begin{Schunk}
\begin{Sinput}
> print(design.table,floating=FALSE)
\end{Sinput}
% latex table generated in R 3.1.1 by xtable 1.7-3 package
% 
\begin{tabular}{rrrrr}
  \hline
 & (Intercept) & sexM & grade & sexM:grade \\ 
  \hline
1 & 1.00 & 1.00 & 6.00 & 6.00 \\ 
  2 & 1.00 & 1.00 & 7.00 & 7.00 \\ 
  3 & 1.00 & 0.00 & 5.00 & 0.00 \\ 
  4 & 1.00 & 1.00 & 3.00 & 3.00 \\ 
  5 & 1.00 & 1.00 & 8.00 & 8.00 \\ 
  6 & 1.00 & 1.00 & 5.00 & 5.00 \\ 
  7 & 1.00 & 0.00 & 8.00 & 0.00 \\ 
  8 & 1.00 & 1.00 & 4.00 & 4.00 \\ 
  9 & 1.00 & 1.00 & 6.00 & 6.00 \\ 
  10 & 1.00 & 1.00 & 7.00 & 7.00 \\ 
   \hline
\end{tabular}\end{Schunk}

\subsection{aov}
\begin{Schunk}
\begin{Sinput}
> fm1 <- aov(tlimth ~ sex + ethnicty + grade + disadvg, data=tli)
> fm1.table <- xtable(fm1)
\end{Sinput}
\end{Schunk}
\begin{Schunk}
\begin{Sinput}
> print(fm1.table,floating=FALSE)
\end{Sinput}
% latex table generated in R 3.1.1 by xtable 1.7-3 package
% 
\begin{tabular}{lrrrrr}
  \hline
 & Df & Sum Sq & Mean Sq & F value & Pr($>$F) \\ 
  \hline
sex & 1 & 75.37 & 75.37 & 0.38 & 0.5417 \\ 
  ethnicty & 3 & 2572.15 & 857.38 & 4.27 & 0.0072 \\ 
  grade & 1 & 36.31 & 36.31 & 0.18 & 0.6717 \\ 
  disadvg & 1 & 59.30 & 59.30 & 0.30 & 0.5882 \\ 
  Residuals & 93 & 18682.87 & 200.89 &  &  \\ 
   \hline
\end{tabular}\end{Schunk}
\subsection{lm}
\begin{Schunk}
\begin{Sinput}
> fm2 <- lm(tlimth ~ sex*ethnicty, data=tli)
> fm2.table <- xtable(fm2)
\end{Sinput}
\end{Schunk}
\begin{Schunk}
\begin{Sinput}
> print(fm2.table,floating=FALSE)
\end{Sinput}
% latex table generated in R 3.1.1 by xtable 1.7-3 package
% 
\begin{tabular}{rrrrr}
  \hline
 & Estimate & Std. Error & t value & Pr($>$$|$t$|$) \\ 
  \hline
(Intercept) & 73.6364 & 4.2502 & 17.33 & 0.0000 \\ 
  sexM & -1.6364 & 5.8842 & -0.28 & 0.7816 \\ 
  ethnictyHISPANIC & -9.7614 & 6.5501 & -1.49 & 0.1395 \\ 
  ethnictyOTHER & 15.8636 & 10.8360 & 1.46 & 0.1466 \\ 
  ethnictyWHITE & 4.7970 & 4.9687 & 0.97 & 0.3368 \\ 
  sexM:ethnictyHISPANIC & 10.6780 & 8.7190 & 1.22 & 0.2238 \\ 
  sexM:ethnictyWHITE & 5.1230 & 7.0140 & 0.73 & 0.4670 \\ 
   \hline
\end{tabular}\end{Schunk}
\subsubsection{anova object}

\begin{Schunk}
\begin{Sinput}
> print(xtable(anova(fm2)),floating=FALSE)
\end{Sinput}
% latex table generated in R 3.1.1 by xtable 1.7-3 package
% 
\begin{tabular}{lrrrrr}
  \hline
 & Df & Sum Sq & Mean Sq & F value & Pr($>$F) \\ 
  \hline
sex & 1 & 75.37 & 75.37 & 0.38 & 0.5395 \\ 
  ethnicty & 3 & 2572.15 & 857.38 & 4.31 & 0.0068 \\ 
  sex:ethnicty & 2 & 298.43 & 149.22 & 0.75 & 0.4748 \\ 
  Residuals & 93 & 18480.04 & 198.71 &  &  \\ 
   \hline
\end{tabular}\end{Schunk}
\subsubsection{Another anova object}
\begin{Schunk}
\begin{Sinput}
> fm2b <- lm(tlimth ~ ethnicty, data=tli)
\end{Sinput}
\end{Schunk}
\begin{Schunk}
\begin{Sinput}
> print(xtable(anova(fm2b,fm2)),floating=FALSE)
\end{Sinput}
% latex table generated in R 3.1.1 by xtable 1.7-3 package
% 
\begin{tabular}{lrrrrrr}
  \hline
 & Res.Df & RSS & Df & Sum of Sq & F & Pr($>$F) \\ 
  \hline
1 & 96 & 19053.59 &  &  &  &  \\ 
  2 & 93 & 18480.04 & 3 & 573.55 & 0.96 & 0.4141 \\ 
   \hline
\end{tabular}\end{Schunk}


\subsection{glm}

\begin{Schunk}
\begin{Sinput}
> ## Demonstrate glm
> fm3 <- glm(disadvg ~ ethnicty*grade, data=tli, family=binomial())
> fm3.table <- xtable(fm3)
\end{Sinput}
\end{Schunk}
\begin{Schunk}
\begin{Sinput}
> print(fm3.table,floating=FALSE)
\end{Sinput}
% latex table generated in R 3.1.1 by xtable 1.7-3 package
% 
\begin{tabular}{rrrrr}
  \hline
 & Estimate & Std. Error & z value & Pr($>$$|$z$|$) \\ 
  \hline
(Intercept) & 3.1888 & 1.5966 & 2.00 & 0.0458 \\ 
  ethnictyHISPANIC & -0.2848 & 2.4808 & -0.11 & 0.9086 \\ 
  ethnictyOTHER & 212.1701 & 22122.7093 & 0.01 & 0.9923 \\ 
  ethnictyWHITE & -8.8150 & 3.3355 & -2.64 & 0.0082 \\ 
  grade & -0.5308 & 0.2892 & -1.84 & 0.0665 \\ 
  ethnictyHISPANIC:grade & 0.2448 & 0.4357 & 0.56 & 0.5742 \\ 
  ethnictyOTHER:grade & -32.6014 & 3393.4687 & -0.01 & 0.9923 \\ 
  ethnictyWHITE:grade & 1.0171 & 0.5185 & 1.96 & 0.0498 \\ 
   \hline
\end{tabular}\end{Schunk}

\subsubsection{anova object}
\begin{Schunk}
\begin{Sinput}
> print(xtable(anova(fm3)),floating=FALSE)
\end{Sinput}
% latex table generated in R 3.1.1 by xtable 1.7-3 package
% 
\begin{tabular}{lrrrr}
  \hline
 & Df & Deviance & Resid. Df & Resid. Dev \\ 
  \hline
NULL &  &  & 99 & 129.49 \\ 
  ethnicty & 3 & 47.24 & 96 & 82.25 \\ 
  grade & 1 & 1.73 & 95 & 80.52 \\ 
  ethnicty:grade & 3 & 7.20 & 92 & 73.32 \\ 
   \hline
\end{tabular}\end{Schunk}


\subsection{More aov}
\begin{Schunk}
\begin{Sinput}
> ## Demonstrate aov
> ## Taken from help(aov) in R 1.1.1
> ## From Venables and Ripley (1997) p.210.
> N <- c(0,1,0,1,1,1,0,0,0,1,1,0,1,1,0,0,1,0,1,0,1,1,0,0)
> P <- c(1,1,0,0,0,1,0,1,1,1,0,0,0,1,0,1,1,0,0,1,0,1,1,0)
> K <- c(1,0,0,1,0,1,1,0,0,1,0,1,0,1,1,0,0,0,1,1,1,0,1,0)
> yield <- c(49.5,62.8,46.8,57.0,59.8,58.5,55.5,56.0,62.8,55.8,69.5,55.0,
+            62.0,48.8,45.5,44.2,52.0,51.5,49.8,48.8,57.2,59.0,53.2,56.0)
> npk <- data.frame(block=gl(6,4), N=factor(N), P=factor(P), K=factor(K), yield=yield)
> npk.aov <- aov(yield ~ block + N*P*K, npk)
> op <- options(contrasts=c("contr.helmert", "contr.treatment"))
> npk.aovE <- aov(yield ~  N*P*K + Error(block), npk)
> options(op)
> #summary(npk.aov)
\end{Sinput}
\end{Schunk}
\begin{Schunk}
\begin{Sinput}
> print(xtable(npk.aov),floating=FALSE)
\end{Sinput}
% latex table generated in R 3.1.1 by xtable 1.7-3 package
% 
\begin{tabular}{lrrrrr}
  \hline
 & Df & Sum Sq & Mean Sq & F value & Pr($>$F) \\ 
  \hline
block & 5 & 343.29 & 68.66 & 4.45 & 0.0159 \\ 
  N & 1 & 189.28 & 189.28 & 12.26 & 0.0044 \\ 
  P & 1 & 8.40 & 8.40 & 0.54 & 0.4749 \\ 
  K & 1 & 95.20 & 95.20 & 6.17 & 0.0288 \\ 
  N:P & 1 & 21.28 & 21.28 & 1.38 & 0.2632 \\ 
  N:K & 1 & 33.13 & 33.13 & 2.15 & 0.1686 \\ 
  P:K & 1 & 0.48 & 0.48 & 0.03 & 0.8628 \\ 
  Residuals & 12 & 185.29 & 15.44 &  &  \\ 
   \hline
\end{tabular}\end{Schunk}

\subsubsection{anova object}
\begin{Schunk}
\begin{Sinput}
> print(xtable(anova(npk.aov)),floating=FALSE)
\end{Sinput}
% latex table generated in R 3.1.1 by xtable 1.7-3 package
% 
\begin{tabular}{lrrrrr}
  \hline
 & Df & Sum Sq & Mean Sq & F value & Pr($>$F) \\ 
  \hline
block & 5 & 343.29 & 68.66 & 4.45 & 0.0159 \\ 
  N & 1 & 189.28 & 189.28 & 12.26 & 0.0044 \\ 
  P & 1 & 8.40 & 8.40 & 0.54 & 0.4749 \\ 
  K & 1 & 95.20 & 95.20 & 6.17 & 0.0288 \\ 
  N:P & 1 & 21.28 & 21.28 & 1.38 & 0.2632 \\ 
  N:K & 1 & 33.13 & 33.13 & 2.15 & 0.1686 \\ 
  P:K & 1 & 0.48 & 0.48 & 0.03 & 0.8628 \\ 
  Residuals & 12 & 185.29 & 15.44 &  &  \\ 
   \hline
\end{tabular}\end{Schunk}

\subsubsection{Another anova object}
\begin{Schunk}
\begin{Sinput}
> print(xtable(summary(npk.aov)),floating=FALSE)
\end{Sinput}
% latex table generated in R 3.1.1 by xtable 1.7-3 package
% 
\begin{tabular}{lrrrrr}
  \hline
 & Df & Sum Sq & Mean Sq & F value & Pr($>$F) \\ 
  \hline
block       & 5 & 343.29 & 68.66 & 4.45 & 0.0159 \\ 
  N           & 1 & 189.28 & 189.28 & 12.26 & 0.0044 \\ 
  P           & 1 & 8.40 & 8.40 & 0.54 & 0.4749 \\ 
  K           & 1 & 95.20 & 95.20 & 6.17 & 0.0288 \\ 
  N:P         & 1 & 21.28 & 21.28 & 1.38 & 0.2632 \\ 
  N:K         & 1 & 33.13 & 33.13 & 2.15 & 0.1686 \\ 
  P:K         & 1 & 0.48 & 0.48 & 0.03 & 0.8628 \\ 
  Residuals   & 12 & 185.29 & 15.44 &  &  \\ 
   \hline
\end{tabular}\end{Schunk}

\begin{Schunk}
\begin{Sinput}
> #summary(npk.aovE)
\end{Sinput}
\end{Schunk}
\begin{Schunk}
\begin{Sinput}
> print(xtable(npk.aovE),floating=FALSE)
\end{Sinput}
% latex table generated in R 3.1.1 by xtable 1.7-3 package
% 
\begin{tabular}{lrrrrr}
  \hline
 & Df & Sum Sq & Mean Sq & F value & Pr($>$F) \\ 
  \hline
N:P:K     & 1 & 37.00 & 37.00 & 0.48 & 0.5252 \\ 
  Residuals & 4 & 306.29 & 76.57 &  &  \\ 
  N         & 1 & 189.28 & 189.28 & 12.26 & 0.0044 \\ 
  P         & 1 & 8.40 & 8.40 & 0.54 & 0.4749 \\ 
  K         & 1 & 95.20 & 95.20 & 6.17 & 0.0288 \\ 
  N:P       & 1 & 21.28 & 21.28 & 1.38 & 0.2632 \\ 
  N:K       & 1 & 33.14 & 33.14 & 2.15 & 0.1686 \\ 
  P:K       & 1 & 0.48 & 0.48 & 0.03 & 0.8628 \\ 
  Residuals1 & 12 & 185.29 & 15.44 &  &  \\ 
   \hline
\end{tabular}\end{Schunk}


\begin{Schunk}
\begin{Sinput}
> print(xtable(summary(npk.aovE)),floating=FALSE)
\end{Sinput}
% latex table generated in R 3.1.1 by xtable 1.7-3 package
% 
\begin{tabular}{lrrrrr}
  \hline
 & Df & Sum Sq & Mean Sq & F value & Pr($>$F) \\ 
  \hline
N:P:K     & 1 & 37.00 & 37.00 & 0.48 & 0.5252 \\ 
  Residuals & 4 & 306.29 & 76.57 &  &  \\ 
  N         & 1 & 189.28 & 189.28 & 12.26 & 0.0044 \\ 
  P         & 1 & 8.40 & 8.40 & 0.54 & 0.4749 \\ 
  K         & 1 & 95.20 & 95.20 & 6.17 & 0.0288 \\ 
  N:P       & 1 & 21.28 & 21.28 & 1.38 & 0.2632 \\ 
  N:K       & 1 & 33.14 & 33.14 & 2.15 & 0.1686 \\ 
  P:K       & 1 & 0.48 & 0.48 & 0.03 & 0.8628 \\ 
  Residuals1 & 12 & 185.29 & 15.44 &  &  \\ 
   \hline
\end{tabular}\end{Schunk}

\subsection{More lm}
\begin{Schunk}
\begin{Sinput}
> ## Demonstrate lm
> ## Taken from help(lm) in R 1.1.1
> ## Annette Dobson (1990) "An Introduction to Generalized Linear Models".
> ## Page 9: Plant Weight Data.
> ctl <- c(4.17,5.58,5.18,6.11,4.50,4.61,5.17,4.53,5.33,5.14)
> trt <- c(4.81,4.17,4.41,3.59,5.87,3.83,6.03,4.89,4.32,4.69)
> group <- gl(2,10,20, labels=c("Ctl","Trt"))
> weight <- c(ctl, trt)
> lm.D9 <- lm(weight ~ group)
\end{Sinput}
\end{Schunk}
\begin{Schunk}
\begin{Sinput}
> print(xtable(lm.D9),floating=FALSE)
\end{Sinput}
% latex table generated in R 3.1.1 by xtable 1.7-3 package
% 
\begin{tabular}{rrrrr}
  \hline
 & Estimate & Std. Error & t value & Pr($>$$|$t$|$) \\ 
  \hline
(Intercept) & 5.0320 & 0.2202 & 22.85 & 0.0000 \\ 
  groupTrt & -0.3710 & 0.3114 & -1.19 & 0.2490 \\ 
   \hline
\end{tabular}\end{Schunk}


\begin{Schunk}
\begin{Sinput}
> print(xtable(anova(lm.D9)),floating=FALSE)
\end{Sinput}
% latex table generated in R 3.1.1 by xtable 1.7-3 package
% 
\begin{tabular}{lrrrrr}
  \hline
 & Df & Sum Sq & Mean Sq & F value & Pr($>$F) \\ 
  \hline
group & 1 & 0.69 & 0.69 & 1.42 & 0.2490 \\ 
  Residuals & 18 & 8.73 & 0.48 &  &  \\ 
   \hline
\end{tabular}\end{Schunk}

\subsection{More glm}
\begin{Schunk}
\begin{Sinput}
> ## Demonstrate glm
> ## Taken from help(glm) in R 1.1.1
> ## Annette Dobson (1990) "An Introduction to Generalized Linear Models".
> ## Page 93: Randomized Controlled Trial :
> counts <- c(18,17,15,20,10,20,25,13,12)
> outcome <- gl(3,1,9)
> treatment <- gl(3,3)
> d.AD <- data.frame(treatment, outcome, counts)
> glm.D93 <- glm(counts ~ outcome + treatment, family=poisson())
\end{Sinput}
\end{Schunk}
\begin{Schunk}
\begin{Sinput}
> print(xtable(glm.D93,align="r|llrc"),floating=FALSE)
\end{Sinput}
% latex table generated in R 3.1.1 by xtable 1.7-3 package
% 
\begin{tabular}{r|llrc}
  \hline
 & Estimate & Std. Error & z value & Pr($>$$|$z$|$) \\ 
  \hline
(Intercept) & 3.0445 & 0.1709 & 17.81 & 0.0000 \\ 
  outcome2 & -0.4543 & 0.2022 & -2.25 & 0.0246 \\ 
  outcome3 & -0.2930 & 0.1927 & -1.52 & 0.1285 \\ 
  treatment2 & 0.0000 & 0.2000 & 0.00 & 1.0000 \\ 
  treatment3 & 0.0000 & 0.2000 & 0.00 & 1.0000 \\ 
   \hline
\end{tabular}\end{Schunk}

\subsection{prcomp}
\begin{Schunk}
\begin{Sinput}
> if(require(stats,quietly=TRUE)) {
+   ## Demonstrate prcomp
+   ## Taken from help(prcomp) in mva package of R 1.1.1
+   data(USArrests)
+   pr1 <- prcomp(USArrests)
+ }
\end{Sinput}
\end{Schunk}
\begin{Schunk}
\begin{Sinput}
> if(require(stats,quietly=TRUE)) {
+   print(xtable(pr1),floating=FALSE)
+ }
\end{Sinput}
% latex table generated in R 3.1.1 by xtable 1.7-3 package
% 
\begin{tabular}{rrrrr}
  \hline
 & PC1 & PC2 & PC3 & PC4 \\ 
  \hline
Murder & 0.0417 & -0.0448 & 0.0799 & -0.9949 \\ 
  Assault & 0.9952 & -0.0588 & -0.0676 & 0.0389 \\ 
  UrbanPop & 0.0463 & 0.9769 & -0.2005 & -0.0582 \\ 
  Rape & 0.0752 & 0.2007 & 0.9741 & 0.0723 \\ 
   \hline
\end{tabular}\end{Schunk}


\begin{Schunk}
\begin{Sinput}
>   print(xtable(summary(pr1)),floating=FALSE)
\end{Sinput}
% latex table generated in R 3.1.1 by xtable 1.7-3 package
% 
\begin{tabular}{rrrrr}
  \hline
 & PC1 & PC2 & PC3 & PC4 \\ 
  \hline
Standard deviation & 83.7324 & 14.2124 & 6.4894 & 2.4828 \\ 
  Proportion of Variance & 0.9655 & 0.0278 & 0.0058 & 0.0008 \\ 
  Cumulative Proportion & 0.9655 & 0.9933 & 0.9991 & 1.0000 \\ 
   \hline
\end{tabular}\end{Schunk}



\begin{Schunk}
\begin{Sinput}
> #  ## Demonstrate princomp
> #  ## Taken from help(princomp) in mva package of R 1.1.1
> #  pr2 <- princomp(USArrests)
> #  print(xtable(pr2))
\end{Sinput}
\end{Schunk}
\subsection{Time series}

\begin{Schunk}
\begin{Sinput}
> data(AirPassengers)
> temp.ts <- ts(AirPassengers, start = c(1949, 1), end = c(1955, 12), frequency = 12)
> temp.table <- xtable(temp.ts,digits=0)
> caption(temp.table) <- "Time series example"
\end{Sinput}
\end{Schunk}
\begin{Schunk}
\begin{Sinput}
> print(temp.table,floating=FALSE)
\end{Sinput}
% latex table generated in R 3.1.1 by xtable 1.7-3 package
% 
\begin{tabular}{rrrrrrrrrrrrr}
  \hline
 & Jan & Feb & Mar & Apr & May & Jun & Jul & Aug & Sep & Oct & Nov & Dec \\ 
  \hline
1949 & 112 & 118 & 132 & 129 & 121 & 135 & 148 & 148 & 136 & 119 & 104 & 118 \\ 
  1950 & 115 & 126 & 141 & 135 & 125 & 149 & 170 & 170 & 158 & 133 & 114 & 140 \\ 
  1951 & 145 & 150 & 178 & 163 & 172 & 178 & 199 & 199 & 184 & 162 & 146 & 166 \\ 
  1952 & 171 & 180 & 193 & 181 & 183 & 218 & 230 & 242 & 209 & 191 & 172 & 194 \\ 
  1953 & 196 & 196 & 236 & 235 & 229 & 243 & 264 & 272 & 237 & 211 & 180 & 201 \\ 
  1954 & 204 & 188 & 235 & 227 & 234 & 264 & 302 & 293 & 259 & 229 & 203 & 229 \\ 
  1955 & 242 & 233 & 267 & 269 & 270 & 315 & 364 & 347 & 312 & 274 & 237 & 278 \\ 
   \hline
\end{tabular}\end{Schunk}

\section{Sanitization}
\begin{Schunk}
\begin{Sinput}
> insane <- data.frame(Name=c("Ampersand","Greater than","Less than","Underscore","Per cent","Dollar","Backslash","Hash", "Caret", "Tilde","Left brace","Right brace"),
+ 				Character = I(c("&",">",		"<",		"_",		"%",		"$",		"\\", "#",	"^",		"~","{","}")))
> colnames(insane)[2] <- paste(insane[,2],collapse="")
\end{Sinput}
\end{Schunk}

\begin{Schunk}
\begin{Sinput}
> print( xtable(insane))
\end{Sinput}
% latex table generated in R 3.1.1 by xtable 1.7-3 package
% 
\begin{table}[ht]
\centering
\begin{tabular}{rll}
  \hline
 & Name & \&$>$$<$\_\%\$$\backslash$\#\verb|^|\~{}\{\} \\ 
  \hline
1 & Ampersand & \& \\ 
  2 & Greater than & $>$ \\ 
  3 & Less than & $<$ \\ 
  4 & Underscore & \_ \\ 
  5 & Per cent & \% \\ 
  6 & Dollar & \$ \\ 
  7 & Backslash & $\backslash$ \\ 
  8 & Hash & \# \\ 
  9 & Caret & \verb|^| \\ 
  10 & Tilde & \~{} \\ 
  11 & Left brace & \{ \\ 
  12 & Right brace & \} \\ 
   \hline
\end{tabular}
\end{table}\end{Schunk}
Sometimes you might want to have your own sanitization function
\begin{Schunk}
\begin{Sinput}
> wanttex <- xtable(data.frame( label=paste("Value_is $10^{-",1:3,"}$",sep="")))
\end{Sinput}
\end{Schunk}
\begin{Schunk}
\begin{Sinput}
> print(wanttex,sanitize.text.function=function(str)gsub("_","\\_",str,fixed=TRUE))
\end{Sinput}
% latex table generated in R 3.1.1 by xtable 1.7-3 package
% 
\begin{table}[ht]
\centering
\begin{tabular}{rl}
  \hline
 & label \\ 
  \hline
1 & Value\_is $10^{-1}$ \\ 
  2 & Value\_is $10^{-2}$ \\ 
  3 & Value\_is $10^{-3}$ \\ 
   \hline
\end{tabular}
\end{table}\end{Schunk}

\subsection{Markup in tables}

Markup can be kept in tables, including column and row names, by using a custom sanitize.text.function:

\begin{Schunk}
\begin{Sinput}
> mat <- round(matrix(c(0.9, 0.89, 200, 0.045, 2.0), c(1, 5)), 4)
> rownames(mat) <- "$y_{t-1}$"
> colnames(mat) <- c("$R^2$", "$\\bar{R}^2$", "F-stat", "S.E.E", "DW")
> mat <- xtable(mat)
\end{Sinput}
\end{Schunk}
\begin{Schunk}
\begin{Sinput}
> print(mat, sanitize.text.function = function(x){x})
\end{Sinput}
% latex table generated in R 3.1.1 by xtable 1.7-3 package
% 
\begin{table}[ht]
\centering
\begin{tabular}{rrrrrr}
  \hline
 & $R^2$ & $\bar{R}^2$ & F-stat & S.E.E & DW \\ 
  \hline
$y_{t-1}$ & 0.90 & 0.89 & 200.00 & 0.04 & 2.00 \\ 
   \hline
\end{tabular}
\end{table}\end{Schunk}

% By David Dahl to demonstrate contribution from David Whitting, 2007-10-09.
You can also have sanitize functions that are specific to column or row names.  In the table below, the row name is not sanitized but column names and table elements are:
\begin{Schunk}
\begin{Sinput}
> money <- matrix(c("$1,000","$900","$100"),ncol=3,dimnames=list("$\\alpha$",c("Income (US$)","Expenses (US$)","Profit (US$)")))
\end{Sinput}
\end{Schunk}
\begin{Schunk}
\begin{Sinput}
> print(xtable(money),sanitize.rownames.function=function(x) {x})
\end{Sinput}
% latex table generated in R 3.1.1 by xtable 1.7-3 package
% 
\begin{table}[ht]
\centering
\begin{tabular}{rlll}
  \hline
 & Income (US\$) & Expenses (US\$) & Profit (US\$) \\ 
  \hline
$\alpha$ & \$1,000 & \$900 & \$100 \\ 
   \hline
\end{tabular}
\end{table}\end{Schunk}

\section{Format examples}
\subsection{Adding a centering environment }
\begin{Schunk}
\begin{Sinput}
>    print(xtable(lm.D9,caption="\\tt latex.environments=NULL"),latex.environments=NULL)
\end{Sinput}
% latex table generated in R 3.1.1 by xtable 1.7-3 package
% 
\begin{table}[ht]
\begin{tabular}{rrrrr}
  \hline
 & Estimate & Std. Error & t value & Pr($>$$|$t$|$) \\ 
  \hline
(Intercept) & 5.0320 & 0.2202 & 22.85 & 0.0000 \\ 
  groupTrt & -0.3710 & 0.3114 & -1.19 & 0.2490 \\ 
   \hline
\end{tabular}
\caption{\tt latex.environments=NULL} 
\end{table}\begin{Sinput}
>     print(xtable(lm.D9,caption="\\tt latex.environments=\"\""),latex.environments="")
\end{Sinput}
% latex table generated in R 3.1.1 by xtable 1.7-3 package
% 
\begin{table}[ht]
\begin{tabular}{rrrrr}
  \hline
 & Estimate & Std. Error & t value & Pr($>$$|$t$|$) \\ 
  \hline
(Intercept) & 5.0320 & 0.2202 & 22.85 & 0.0000 \\ 
  groupTrt & -0.3710 & 0.3114 & -1.19 & 0.2490 \\ 
   \hline
\end{tabular}
\caption{\tt latex.environments=""} 
\end{table}\begin{Sinput}
>     print(xtable(lm.D9,caption="\\tt latex.environments=\"center\""),latex.environments="center")
\end{Sinput}
% latex table generated in R 3.1.1 by xtable 1.7-3 package
% 
\begin{table}[ht]
\centering
\begin{tabular}{rrrrr}
  \hline
 & Estimate & Std. Error & t value & Pr($>$$|$t$|$) \\ 
  \hline
(Intercept) & 5.0320 & 0.2202 & 22.85 & 0.0000 \\ 
  groupTrt & -0.3710 & 0.3114 & -1.19 & 0.2490 \\ 
   \hline
\end{tabular}
\caption{\tt latex.environments="center"} 
\end{table}\end{Schunk}
\subsection{Column alignment}

\begin{Schunk}
\begin{Sinput}
> tli.table <- xtable(tli[1:10,])
\end{Sinput}
\end{Schunk}
\begin{Schunk}
\begin{Sinput}
> align(tli.table) <- rep("r",6)
\end{Sinput}
\end{Schunk}
\begin{Schunk}
\begin{Sinput}
> print(tli.table,floating=FALSE)
\end{Sinput}
% latex table generated in R 3.1.1 by xtable 1.7-3 package
% 
\begin{tabular}{rrrrrr}
  \hline
 & grade & sex & disadvg & ethnicty & tlimth \\ 
  \hline
1 &   6 & M & YES & HISPANIC &  43 \\ 
  2 &   7 & M & NO & BLACK &  88 \\ 
  3 &   5 & F & YES & HISPANIC &  34 \\ 
  4 &   3 & M & YES & HISPANIC &  65 \\ 
  5 &   8 & M & YES & WHITE &  75 \\ 
  6 &   5 & M & NO & BLACK &  74 \\ 
  7 &   8 & F & YES & HISPANIC &  72 \\ 
  8 &   4 & M & YES & BLACK &  79 \\ 
  9 &   6 & M & NO & WHITE &  88 \\ 
  10 &   7 & M & YES & HISPANIC &  87 \\ 
   \hline
\end{tabular}\end{Schunk}
\subsubsection{Single string and column lines}
\begin{Schunk}
\begin{Sinput}
> align(tli.table) <- "|rrl|l|lr|"
\end{Sinput}
\end{Schunk}
\begin{Schunk}
\begin{Sinput}
> print(tli.table,floating=FALSE)
\end{Sinput}
% latex table generated in R 3.1.1 by xtable 1.7-3 package
% 
\begin{tabular}{|rrl|l|lr|}
  \hline
 & grade & sex & disadvg & ethnicty & tlimth \\ 
  \hline
1 &   6 & M & YES & HISPANIC &  43 \\ 
  2 &   7 & M & NO & BLACK &  88 \\ 
  3 &   5 & F & YES & HISPANIC &  34 \\ 
  4 &   3 & M & YES & HISPANIC &  65 \\ 
  5 &   8 & M & YES & WHITE &  75 \\ 
  6 &   5 & M & NO & BLACK &  74 \\ 
  7 &   8 & F & YES & HISPANIC &  72 \\ 
  8 &   4 & M & YES & BLACK &  79 \\ 
  9 &   6 & M & NO & WHITE &  88 \\ 
  10 &   7 & M & YES & HISPANIC &  87 \\ 
   \hline
\end{tabular}\end{Schunk}
\subsubsection{Fixed width columns}
\begin{Schunk}
\begin{Sinput}
> align(tli.table) <- "|rr|lp{3cm}l|r|"
\end{Sinput}
\end{Schunk}
\begin{Schunk}
\begin{Sinput}
> print(tli.table,floating=FALSE)
\end{Sinput}
% latex table generated in R 3.1.1 by xtable 1.7-3 package
% 
\begin{tabular}{|rr|lp{3cm}l|r|}
  \hline
 & grade & sex & disadvg & ethnicty & tlimth \\ 
  \hline
1 &   6 & M & YES & HISPANIC &  43 \\ 
  2 &   7 & M & NO & BLACK &  88 \\ 
  3 &   5 & F & YES & HISPANIC &  34 \\ 
  4 &   3 & M & YES & HISPANIC &  65 \\ 
  5 &   8 & M & YES & WHITE &  75 \\ 
  6 &   5 & M & NO & BLACK &  74 \\ 
  7 &   8 & F & YES & HISPANIC &  72 \\ 
  8 &   4 & M & YES & BLACK &  79 \\ 
  9 &   6 & M & NO & WHITE &  88 \\ 
  10 &   7 & M & YES & HISPANIC &  87 \\ 
   \hline
\end{tabular}\end{Schunk}

\subsection{Significant digits}


Specify with a single argument
\begin{Schunk}
\begin{Sinput}
> digits(tli.table) <- 3
\end{Sinput}
\end{Schunk}
\begin{Schunk}
\begin{Sinput}
> print(tli.table,floating=FALSE,)
\end{Sinput}
% latex table generated in R 3.1.1 by xtable 1.7-3 package
% 
\begin{tabular}{|rr|lp{3cm}l|r|}
  \hline
 & grade & sex & disadvg & ethnicty & tlimth \\ 
  \hline
1 &    6 & M & YES & HISPANIC &   43 \\ 
  2 &    7 & M & NO & BLACK &   88 \\ 
  3 &    5 & F & YES & HISPANIC &   34 \\ 
  4 &    3 & M & YES & HISPANIC &   65 \\ 
  5 &    8 & M & YES & WHITE &   75 \\ 
  6 &    5 & M & NO & BLACK &   74 \\ 
  7 &    8 & F & YES & HISPANIC &   72 \\ 
  8 &    4 & M & YES & BLACK &   79 \\ 
  9 &    6 & M & NO & WHITE &   88 \\ 
  10 &    7 & M & YES & HISPANIC &   87 \\ 
   \hline
\end{tabular}\end{Schunk}


or one for each column, counting the row names
\begin{Schunk}
\begin{Sinput}
> digits(tli.table) <- 1:(ncol(tli)+1)
\end{Sinput}
\end{Schunk}
\begin{Schunk}
\begin{Sinput}
> print(tli.table,floating=FALSE,)
\end{Sinput}
% latex table generated in R 3.1.1 by xtable 1.7-3 package
% 
\begin{tabular}{|rr|lp{3cm}l|r|}
  \hline
 & grade & sex & disadvg & ethnicty & tlimth \\ 
  \hline
1 &   6 & M & YES & HISPANIC &      43 \\ 
  2 &   7 & M & NO & BLACK &      88 \\ 
  3 &   5 & F & YES & HISPANIC &      34 \\ 
  4 &   3 & M & YES & HISPANIC &      65 \\ 
  5 &   8 & M & YES & WHITE &      75 \\ 
  6 &   5 & M & NO & BLACK &      74 \\ 
  7 &   8 & F & YES & HISPANIC &      72 \\ 
  8 &   4 & M & YES & BLACK &      79 \\ 
  9 &   6 & M & NO & WHITE &      88 \\ 
  10 &   7 & M & YES & HISPANIC &      87 \\ 
   \hline
\end{tabular}\end{Schunk}


or as a full matrix
\begin{Schunk}
\begin{Sinput}
> digits(tli.table) <- matrix( 0:4, nrow = 10, ncol = ncol(tli)+1 )
\end{Sinput}
\end{Schunk}
\begin{Schunk}
\begin{Sinput}
> print(tli.table,floating=FALSE,)
\end{Sinput}
% latex table generated in R 3.1.1 by xtable 1.7-3 package
% 
\begin{tabular}{|rr|lp{3cm}l|r|}
  \hline
 & grade & sex & disadvg & ethnicty & tlimth \\ 
  \hline
1 & 6 & M & YES & HISPANIC & 43 \\ 
  2 &  7 & M & NO & BLACK & 88 \\ 
  3 &   5 & F & YES & HISPANIC &  34 \\ 
  4 &    3 & M & YES & HISPANIC &   65 \\ 
  5 &     8 & M & YES & WHITE &    75 \\ 
  6 & 5 & M & NO & BLACK & 74 \\ 
  7 &  8 & F & YES & HISPANIC & 72 \\ 
  8 &   4 & M & YES & BLACK &  79 \\ 
  9 &    6 & M & NO & WHITE &   88 \\ 
  10 &     7 & M & YES & HISPANIC &    87 \\ 
   \hline
\end{tabular}\end{Schunk}

\subsection{Suppress row names}
\begin{Schunk}
\begin{Sinput}
> print((tli.table),include.rownames=FALSE,floating=FALSE)
\end{Sinput}
% latex table generated in R 3.1.1 by xtable 1.7-3 package
% 
\begin{tabular}{r|lp{3cm}l|r|}
  \hline
grade & sex & disadvg & ethnicty & tlimth \\ 
  \hline
6 & M & YES & HISPANIC & 43 \\ 
   7 & M & NO & BLACK & 88 \\ 
    5 & F & YES & HISPANIC &  34 \\ 
     3 & M & YES & HISPANIC &   65 \\ 
      8 & M & YES & WHITE &    75 \\ 
  5 & M & NO & BLACK & 74 \\ 
   8 & F & YES & HISPANIC & 72 \\ 
    4 & M & YES & BLACK &  79 \\ 
     6 & M & NO & WHITE &   88 \\ 
      7 & M & YES & HISPANIC &    87 \\ 
   \hline
\end{tabular}\end{Schunk}

If you want a vertical line on the left, you need to change the align attribute.
\begin{Schunk}
\begin{Sinput}
> align(tli.table) <- "|r|r|lp{3cm}l|r|"
\end{Sinput}
\end{Schunk}
\begin{Schunk}
\begin{Sinput}
> print((tli.table),include.rownames=FALSE,floating=FALSE)
\end{Sinput}
% latex table generated in R 3.1.1 by xtable 1.7-3 package
% 
\begin{tabular}{|r|lp{3cm}l|r|}
  \hline
grade & sex & disadvg & ethnicty & tlimth \\ 
  \hline
6 & M & YES & HISPANIC & 43 \\ 
   7 & M & NO & BLACK & 88 \\ 
    5 & F & YES & HISPANIC &  34 \\ 
     3 & M & YES & HISPANIC &   65 \\ 
      8 & M & YES & WHITE &    75 \\ 
  5 & M & NO & BLACK & 74 \\ 
   8 & F & YES & HISPANIC & 72 \\ 
    4 & M & YES & BLACK &  79 \\ 
     6 & M & NO & WHITE &   88 \\ 
      7 & M & YES & HISPANIC &    87 \\ 
   \hline
\end{tabular}\end{Schunk}

Revert the alignment to what is was before.
\begin{Schunk}
\begin{Sinput}
> align(tli.table) <- "|rr|lp{3cm}l|r|"
\end{Sinput}
\end{Schunk}

\subsection{Suppress column names}
\begin{Schunk}
\begin{Sinput}
> print((tli.table),include.colnames=FALSE,floating=FALSE)
\end{Sinput}
% latex table generated in R 3.1.1 by xtable 1.7-3 package
% 
\begin{tabular}{|rr|lp{3cm}l|r|}
  \hline
  \hline
1 & 6 & M & YES & HISPANIC & 43 \\ 
  2 &  7 & M & NO & BLACK & 88 \\ 
  3 &   5 & F & YES & HISPANIC &  34 \\ 
  4 &    3 & M & YES & HISPANIC &   65 \\ 
  5 &     8 & M & YES & WHITE &    75 \\ 
  6 & 5 & M & NO & BLACK & 74 \\ 
  7 &  8 & F & YES & HISPANIC & 72 \\ 
  8 &   4 & M & YES & BLACK &  79 \\ 
  9 &    6 & M & NO & WHITE &   88 \\ 
  10 &     7 & M & YES & HISPANIC &    87 \\ 
   \hline
\end{tabular}\end{Schunk}
\\
Note the doubled header lines which can be suppressed with, eg,
\begin{Schunk}
\begin{Sinput}
> print(tli.table,include.colnames=FALSE,floating=FALSE,hline.after=c(0,nrow(tli.table)))
\end{Sinput}
% latex table generated in R 3.1.1 by xtable 1.7-3 package
% 
\begin{tabular}{|rr|lp{3cm}l|r|}
   \hline
1 & 6 & M & YES & HISPANIC & 43 \\ 
  2 &  7 & M & NO & BLACK & 88 \\ 
  3 &   5 & F & YES & HISPANIC &  34 \\ 
  4 &    3 & M & YES & HISPANIC &   65 \\ 
  5 &     8 & M & YES & WHITE &    75 \\ 
  6 & 5 & M & NO & BLACK & 74 \\ 
  7 &  8 & F & YES & HISPANIC & 72 \\ 
  8 &   4 & M & YES & BLACK &  79 \\ 
  9 &    6 & M & NO & WHITE &   88 \\ 
  10 &     7 & M & YES & HISPANIC &    87 \\ 
   \hline
\end{tabular}\end{Schunk}

\subsection{Suppress row and column names}
\begin{Schunk}
\begin{Sinput}
> print((tli.table),include.colnames=FALSE,include.rownames=FALSE,floating=FALSE)
\end{Sinput}
% latex table generated in R 3.1.1 by xtable 1.7-3 package
% 
\begin{tabular}{r|lp{3cm}l|r|}
  \hline
  \hline
6 & M & YES & HISPANIC & 43 \\ 
   7 & M & NO & BLACK & 88 \\ 
    5 & F & YES & HISPANIC &  34 \\ 
     3 & M & YES & HISPANIC &   65 \\ 
      8 & M & YES & WHITE &    75 \\ 
  5 & M & NO & BLACK & 74 \\ 
   8 & F & YES & HISPANIC & 72 \\ 
    4 & M & YES & BLACK &  79 \\ 
     6 & M & NO & WHITE &   88 \\ 
      7 & M & YES & HISPANIC &    87 \\ 
   \hline
\end{tabular}\end{Schunk}

\subsection{Rotate row and column names}
The {\tt rotate.rownames } and {\tt rotate.colnames} arguments can be
used to rotate the row and/or column names.

\begin{Schunk}
\begin{Sinput}
> print((tli.table),rotate.rownames=TRUE,rotate.colnames=TRUE)
\end{Sinput}
% latex table generated in R 3.1.1 by xtable 1.7-3 package
% 
\begin{table}[ht]
\centering
\begin{tabular}{|rr|lp{3cm}l|r|}
  \hline
 & \begin{sideways} grade \end{sideways} & \begin{sideways} sex \end{sideways} & \begin{sideways} disadvg \end{sideways} & \begin{sideways} ethnicty \end{sideways} & \begin{sideways} tlimth \end{sideways} \\ 
  \hline
\begin{sideways} 1 \end{sideways} & 6 & M & YES & HISPANIC & 43 \\ 
  \begin{sideways} 2 \end{sideways} &  7 & M & NO & BLACK & 88 \\ 
  \begin{sideways} 3 \end{sideways} &   5 & F & YES & HISPANIC &  34 \\ 
  \begin{sideways} 4 \end{sideways} &    3 & M & YES & HISPANIC &   65 \\ 
  \begin{sideways} 5 \end{sideways} &     8 & M & YES & WHITE &    75 \\ 
  \begin{sideways} 6 \end{sideways} & 5 & M & NO & BLACK & 74 \\ 
  \begin{sideways} 7 \end{sideways} &  8 & F & YES & HISPANIC & 72 \\ 
  \begin{sideways} 8 \end{sideways} &   4 & M & YES & BLACK &  79 \\ 
  \begin{sideways} 9 \end{sideways} &    6 & M & NO & WHITE &   88 \\ 
  \begin{sideways} 10 \end{sideways} &     7 & M & YES & HISPANIC &    87 \\ 
   \hline
\end{tabular}
\end{table}\end{Schunk}

\subsection{Horizontal lines}

\subsubsection{Line locations}

Use the {\tt hline.after} argument to specify the position of the horizontal lines.

\begin{Schunk}
\begin{Sinput}
> print(xtable(anova(glm.D93)),hline.after=c(1),floating=FALSE)
\end{Sinput}
% latex table generated in R 3.1.1 by xtable 1.7-3 package
% 
\begin{tabular}{lrrrr}
  & Df & Deviance & Resid. Df & Resid. Dev \\ 
 NULL &  &  & 8 & 10.58 \\ 
   \hline
outcome & 2 & 5.45 & 6 & 5.13 \\ 
  treatment & 2 & 0.00 & 4 & 5.13 \\ 
  \end{tabular}\end{Schunk}

\subsubsection{Line styles}

The \LaTeX package {\tt booktabs} can be used to specify different
line style tags for top, middle, and bottom lines.  Specifying
{\tt booktabs = TRUE} will lead to separate tags being generated
for the three line types.

Insert \verb|\usepackage{booktabs}| in your \LaTeX preamble and define
the {\tt toprule}, {\tt midrule}, and {\tt bottomrule} tags to specify
the line styles. By default, when no value is given for
\texttt{hline.after}, a \texttt{toprule} will be drawn above the
table, a \texttt{midrule} after the table headings and a
\texttt{bottomrule} below the table. The width of the top and bottom
rules can be set by supplying a value to \verb+\heavyrulewidth+. The
width of the midrules can be set by supplying a value to
\verb+\lightrulewidth+. The following tables have
\verb+\heavyrulewidth = 2pt+ and \verb+\lightrulewidth = 0.5pt+, to
ensure the difference in weight is noticeable.

There is no support for \verb+\cmidrule+ or \verb+\specialrule+
although they are part of the \texttt{booktabs} package.

\heavyrulewidth = 2pt
\lightrulewidth = 0.5pt

\begin{Schunk}
\begin{Sinput}
> print(tli.table, booktabs=TRUE, floating = FALSE)
\end{Sinput}
% latex table generated in R 3.1.1 by xtable 1.7-3 package
% 
\begin{tabular}{|rr|lp{3cm}l|r|}
  \toprule
 & grade & sex & disadvg & ethnicty & tlimth \\ 
  \midrule
1 & 6 & M & YES & HISPANIC & 43 \\ 
  2 &  7 & M & NO & BLACK & 88 \\ 
  3 &   5 & F & YES & HISPANIC &  34 \\ 
  4 &    3 & M & YES & HISPANIC &   65 \\ 
  5 &     8 & M & YES & WHITE &    75 \\ 
  6 & 5 & M & NO & BLACK & 74 \\ 
  7 &  8 & F & YES & HISPANIC & 72 \\ 
  8 &   4 & M & YES & BLACK &  79 \\ 
  9 &    6 & M & NO & WHITE &   88 \\ 
  10 &     7 & M & YES & HISPANIC &    87 \\ 
   \bottomrule
\end{tabular}\end{Schunk}

\vspace{12pt}
If \texttt{hline.after} includes $-1$, a \texttt{toprule} will be
drawn above the table. If \texttt{hline.after} includes the number of
rows in the table, a \texttt{bottomrule} will be drawn below the
table. For any other values specified in \texttt{hline.after}, a
\texttt{midrule} will be drawn after that line of the table.

The next table has more than one \texttt{midrule}.

\begin{Schunk}
\begin{Sinput}
> bktbs <- xtable(matrix(1:10, ncol = 2))
> hlines <- c(-1,0,1,nrow(bktbs))
\end{Sinput}
\end{Schunk}
This command produces the required table.
\begin{Schunk}
\begin{Sinput}
> print(bktbs, booktabs = TRUE, hline.after = hlines, floating = FALSE)
\end{Sinput}
% latex table generated in R 3.1.1 by xtable 1.7-3 package
% 
\begin{tabular}{rrr}
  \toprule
 & 1 & 2 \\ 
  \midrule
1 &   1 &   6 \\ 
   \midrule
2 &   2 &   7 \\ 
  3 &   3 &   8 \\ 
  4 &   4 &   9 \\ 
  5 &   5 &  10 \\ 
   \bottomrule
\end{tabular}\end{Schunk}


\subsection{Table-level \LaTeX}
\begin{Schunk}
\begin{Sinput}
> print(xtable(anova(glm.D93)),size="small",floating=FALSE)
\end{Sinput}
% latex table generated in R 3.1.1 by xtable 1.7-3 package
% 
{\small
\begin{tabular}{lrrrr}
  \hline
 & Df & Deviance & Resid. Df & Resid. Dev \\ 
  \hline
NULL &  &  & 8 & 10.58 \\ 
  outcome & 2 & 5.45 & 6 & 5.13 \\ 
  treatment & 2 & 0.00 & 4 & 5.13 \\ 
   \hline
\end{tabular}
}\end{Schunk}


\subsection{Long tables}
Remember to insert \verb|\usepackage{longtable}| in your \LaTeX preamble.

\begin{Schunk}
\begin{Sinput}
> ## Demonstration of longtable support.
> data(Seatbelts)
> x <- Seatbelts[0:100,]; x <- x[,-6]
> x.big <- xtable(x,label='tabbig',
+ 	caption='Example of longtable spanning several pages')
> digits(x.big) <- 0
\end{Sinput}
\end{Schunk}
\begin{Schunk}
\begin{Sinput}
> print(x.big,tabular.environment='longtable',floating=FALSE)
\end{Sinput}
% latex table generated in R 3.1.1 by xtable 1.7-3 package
% 
\begin{longtable}{rrrrrrrr}
  \hline
 & DriversKilled & drivers & front & rear & kms & VanKilled & law \\ 
  \hline
1 & 107 & 1687 & 867 & 269 & 9059 & 12 & 0 \\ 
  2 & 97 & 1508 & 825 & 265 & 7685 & 6 & 0 \\ 
  3 & 102 & 1507 & 806 & 319 & 9963 & 12 & 0 \\ 
  4 & 87 & 1385 & 814 & 407 & 10955 & 8 & 0 \\ 
  5 & 119 & 1632 & 991 & 454 & 11823 & 10 & 0 \\ 
  6 & 106 & 1511 & 945 & 427 & 12391 & 13 & 0 \\ 
  7 & 110 & 1559 & 1004 & 522 & 13460 & 11 & 0 \\ 
  8 & 106 & 1630 & 1091 & 536 & 14055 & 6 & 0 \\ 
  9 & 107 & 1579 & 958 & 405 & 12106 & 10 & 0 \\ 
  10 & 134 & 1653 & 850 & 437 & 11372 & 16 & 0 \\ 
  11 & 147 & 2152 & 1109 & 434 & 9834 & 13 & 0 \\ 
  12 & 180 & 2148 & 1113 & 437 & 9267 & 14 & 0 \\ 
  13 & 125 & 1752 & 925 & 316 & 9130 & 14 & 0 \\ 
  14 & 134 & 1765 & 903 & 311 & 8933 & 6 & 0 \\ 
  15 & 110 & 1717 & 1006 & 351 & 11000 & 8 & 0 \\ 
  16 & 102 & 1558 & 892 & 362 & 10733 & 11 & 0 \\ 
  17 & 103 & 1575 & 990 & 486 & 12912 & 7 & 0 \\ 
  18 & 111 & 1520 & 866 & 429 & 12926 & 13 & 0 \\ 
  19 & 120 & 1805 & 1095 & 551 & 13990 & 13 & 0 \\ 
  20 & 129 & 1800 & 1204 & 646 & 14926 & 11 & 0 \\ 
  21 & 122 & 1719 & 1029 & 456 & 12900 & 11 & 0 \\ 
  22 & 183 & 2008 & 1147 & 475 & 12034 & 14 & 0 \\ 
  23 & 169 & 2242 & 1171 & 456 & 10643 & 16 & 0 \\ 
  24 & 190 & 2478 & 1299 & 468 & 10742 & 14 & 0 \\ 
  25 & 134 & 2030 & 944 & 356 & 10266 & 17 & 0 \\ 
  26 & 108 & 1655 & 874 & 271 & 10281 & 16 & 0 \\ 
  27 & 104 & 1693 & 840 & 354 & 11527 & 15 & 0 \\ 
  28 & 117 & 1623 & 893 & 427 & 12281 & 13 & 0 \\ 
  29 & 157 & 1805 & 1007 & 465 & 13587 & 13 & 0 \\ 
  30 & 148 & 1746 & 973 & 440 & 13049 & 15 & 0 \\ 
  31 & 130 & 1795 & 1097 & 539 & 16055 & 12 & 0 \\ 
  32 & 140 & 1926 & 1194 & 646 & 15220 & 6 & 0 \\ 
  33 & 136 & 1619 & 988 & 457 & 13824 & 9 & 0 \\ 
  34 & 140 & 1992 & 1077 & 446 & 12729 & 13 & 0 \\ 
  35 & 187 & 2233 & 1045 & 402 & 11467 & 14 & 0 \\ 
  36 & 150 & 2192 & 1115 & 441 & 11351 & 15 & 0 \\ 
  37 & 159 & 2080 & 1005 & 359 & 10803 & 14 & 0 \\ 
  38 & 143 & 1768 & 857 & 334 & 10548 & 3 & 0 \\ 
  39 & 114 & 1835 & 879 & 312 & 12368 & 12 & 0 \\ 
  40 & 127 & 1569 & 887 & 427 & 13311 & 13 & 0 \\ 
  41 & 159 & 1976 & 1075 & 434 & 13885 & 12 & 0 \\ 
  42 & 156 & 1853 & 1121 & 486 & 14088 & 8 & 0 \\ 
  43 & 138 & 1965 & 1190 & 569 & 16932 & 8 & 0 \\ 
  44 & 120 & 1689 & 1058 & 523 & 16164 & 15 & 0 \\ 
  45 & 117 & 1778 & 939 & 418 & 14883 & 8 & 0 \\ 
  46 & 170 & 1976 & 1074 & 452 & 13532 & 5 & 0 \\ 
  47 & 168 & 2397 & 1089 & 462 & 12220 & 17 & 0 \\ 
  48 & 198 & 2654 & 1208 & 497 & 12025 & 14 & 0 \\ 
  49 & 144 & 2097 & 903 & 354 & 11692 & 13 & 0 \\ 
  50 & 146 & 1963 & 916 & 347 & 11081 & 5 & 0 \\ 
  51 & 109 & 1677 & 787 & 276 & 13745 & 8 & 0 \\ 
  52 & 131 & 1941 & 1114 & 472 & 14382 & 5 & 0 \\ 
  53 & 151 & 2003 & 1014 & 487 & 14391 & 12 & 0 \\ 
  54 & 140 & 1813 & 1022 & 505 & 15597 & 11 & 0 \\ 
  55 & 153 & 2012 & 1114 & 619 & 16834 & 13 & 0 \\ 
  56 & 140 & 1912 & 1132 & 640 & 17282 & 15 & 0 \\ 
  57 & 161 & 2084 & 1111 & 559 & 15779 & 11 & 0 \\ 
  58 & 168 & 2080 & 1008 & 453 & 13946 & 11 & 0 \\ 
  59 & 152 & 2118 & 916 & 418 & 12701 & 10 & 0 \\ 
  60 & 136 & 2150 & 992 & 419 & 10431 & 13 & 0 \\ 
  61 & 113 & 1608 & 731 & 262 & 11616 & 8 & 0 \\ 
  62 & 100 & 1503 & 665 & 299 & 10808 & 6 & 0 \\ 
  63 & 103 & 1548 & 724 & 303 & 12421 & 8 & 0 \\ 
  64 & 103 & 1382 & 744 & 401 & 13605 & 14 & 0 \\ 
  65 & 121 & 1731 & 910 & 413 & 14455 & 12 & 0 \\ 
  66 & 134 & 1798 & 883 & 426 & 15019 & 14 & 0 \\ 
  67 & 133 & 1779 & 900 & 516 & 15662 & 13 & 0 \\ 
  68 & 129 & 1887 & 1057 & 600 & 16745 & 9 & 0 \\ 
  69 & 144 & 2004 & 1076 & 459 & 14717 & 4 & 0 \\ 
  70 & 154 & 2077 & 919 & 443 & 13756 & 13 & 0 \\ 
  71 & 156 & 2092 & 920 & 412 & 12531 & 6 & 0 \\ 
  72 & 163 & 2051 & 953 & 400 & 12568 & 15 & 0 \\ 
  73 & 122 & 1577 & 664 & 278 & 11249 & 12 & 0 \\ 
  74 & 92 & 1356 & 607 & 302 & 11096 & 16 & 0 \\ 
  75 & 117 & 1652 & 777 & 381 & 12637 & 7 & 0 \\ 
  76 & 95 & 1382 & 633 & 279 & 13018 & 12 & 0 \\ 
  77 & 96 & 1519 & 791 & 442 & 15005 & 10 & 0 \\ 
  78 & 108 & 1421 & 790 & 409 & 15235 & 9 & 0 \\ 
  79 & 108 & 1442 & 803 & 416 & 15552 & 9 & 0 \\ 
  80 & 106 & 1543 & 884 & 511 & 16905 & 6 & 0 \\ 
  81 & 140 & 1656 & 769 & 393 & 14776 & 7 & 0 \\ 
  82 & 114 & 1561 & 732 & 345 & 14104 & 13 & 0 \\ 
  83 & 158 & 1905 & 859 & 391 & 12854 & 14 & 0 \\ 
  84 & 161 & 2199 & 994 & 470 & 12956 & 13 & 0 \\ 
  85 & 102 & 1473 & 704 & 266 & 12177 & 14 & 0 \\ 
  86 & 127 & 1655 & 684 & 312 & 11918 & 11 & 0 \\ 
  87 & 125 & 1407 & 671 & 300 & 13517 & 11 & 0 \\ 
  88 & 101 & 1395 & 643 & 373 & 14417 & 10 & 0 \\ 
  89 & 97 & 1530 & 771 & 412 & 15911 & 4 & 0 \\ 
  90 & 112 & 1309 & 644 & 322 & 15589 & 8 & 0 \\ 
  91 & 112 & 1526 & 828 & 458 & 16543 & 9 & 0 \\ 
  92 & 113 & 1327 & 748 & 427 & 17925 & 10 & 0 \\ 
  93 & 108 & 1627 & 767 & 346 & 15406 & 10 & 0 \\ 
  94 & 128 & 1748 & 825 & 421 & 14601 & 5 & 0 \\ 
  95 & 154 & 1958 & 810 & 344 & 13107 & 13 & 0 \\ 
  96 & 162 & 2274 & 986 & 370 & 12268 & 12 & 0 \\ 
  97 & 112 & 1648 & 714 & 291 & 11972 & 10 & 0 \\ 
  98 & 79 & 1401 & 567 & 224 & 12028 & 9 & 0 \\ 
  99 & 82 & 1411 & 616 & 266 & 14033 & 7 & 0 \\ 
  100 & 127 & 1403 & 678 & 338 & 14244 & 5 & 0 \\ 
   \hline
\hline
\caption{Example of longtable spanning several pages} 
\label{tabbig}
\end{longtable}\end{Schunk}

%%
%% The column name alignment is off in the following example.
%% It needs some revision before exposing it. - CR, 7/2/2012
%%
%
%\subsubsection{Long tables with the header on each page}
%
%The {\tt add.to.row} argument can be used to display the header
%for a long table on each page, and to add a "continued" footer
%on all pages except the last page.
%
%<<results=tex>>=
%library(xtable)
%x<-matrix(rnorm(1000), ncol = 10)
%addtorow<-list()
%addtorow$pos<-list()
%addtorow$pos[[1]]<-c(0)
%addtorow$command<-c(paste(
%    "\\hline \n",
%    "  \\endhead \n",
%    "  \\hline \n",
%    "  {\\footnotesize Continued on next page} \n",
%    "  \\endfoot \n",
%    "  \\endlastfoot \n",sep=""))
%x.big2 <- xtable(x, label = "tabbig2",
%    caption = "Example of longtable with the header on each page")
%print(x.big2, tabular.environment = "longtable", floating = FALSE,
%include.rownames=FALSE, add.to.row=addtorow, hline.after=c(-1) )
%@

\subsection{Sideways tables}
Remember to insert \verb|\usepackage{rotating}| in your LaTeX preamble.
Sideways tables can't be forced in place with the `H' specifier, but you can
use the \verb|\clearpage| command to get them fairly nearby.

\begin{Schunk}
\begin{Sinput}
> x <- x[1:30,]
> x.small <- xtable(x,label='tabsmall',caption='A sideways table', digits = 0)
\end{Sinput}
\end{Schunk}

\begin{Schunk}
\begin{Sinput}
> print(x.small,floating.environment='sidewaystable')
\end{Sinput}
% latex table generated in R 3.1.1 by xtable 1.7-3 package
% 
\begin{sidewaystable}[ht]
\centering
\begin{tabular}{rrrrrrrr}
  \hline
 & DriversKilled & drivers & front & rear & kms & VanKilled & law \\ 
  \hline
1 & 107 & 1687 & 867 & 269 & 9059 & 12 & 0 \\ 
  2 & 97 & 1508 & 825 & 265 & 7685 & 6 & 0 \\ 
  3 & 102 & 1507 & 806 & 319 & 9963 & 12 & 0 \\ 
  4 & 87 & 1385 & 814 & 407 & 10955 & 8 & 0 \\ 
  5 & 119 & 1632 & 991 & 454 & 11823 & 10 & 0 \\ 
  6 & 106 & 1511 & 945 & 427 & 12391 & 13 & 0 \\ 
  7 & 110 & 1559 & 1004 & 522 & 13460 & 11 & 0 \\ 
  8 & 106 & 1630 & 1091 & 536 & 14055 & 6 & 0 \\ 
  9 & 107 & 1579 & 958 & 405 & 12106 & 10 & 0 \\ 
  10 & 134 & 1653 & 850 & 437 & 11372 & 16 & 0 \\ 
  11 & 147 & 2152 & 1109 & 434 & 9834 & 13 & 0 \\ 
  12 & 180 & 2148 & 1113 & 437 & 9267 & 14 & 0 \\ 
  13 & 125 & 1752 & 925 & 316 & 9130 & 14 & 0 \\ 
  14 & 134 & 1765 & 903 & 311 & 8933 & 6 & 0 \\ 
  15 & 110 & 1717 & 1006 & 351 & 11000 & 8 & 0 \\ 
  16 & 102 & 1558 & 892 & 362 & 10733 & 11 & 0 \\ 
  17 & 103 & 1575 & 990 & 486 & 12912 & 7 & 0 \\ 
  18 & 111 & 1520 & 866 & 429 & 12926 & 13 & 0 \\ 
  19 & 120 & 1805 & 1095 & 551 & 13990 & 13 & 0 \\ 
  20 & 129 & 1800 & 1204 & 646 & 14926 & 11 & 0 \\ 
  21 & 122 & 1719 & 1029 & 456 & 12900 & 11 & 0 \\ 
  22 & 183 & 2008 & 1147 & 475 & 12034 & 14 & 0 \\ 
  23 & 169 & 2242 & 1171 & 456 & 10643 & 16 & 0 \\ 
  24 & 190 & 2478 & 1299 & 468 & 10742 & 14 & 0 \\ 
  25 & 134 & 2030 & 944 & 356 & 10266 & 17 & 0 \\ 
  26 & 108 & 1655 & 874 & 271 & 10281 & 16 & 0 \\ 
  27 & 104 & 1693 & 840 & 354 & 11527 & 15 & 0 \\ 
  28 & 117 & 1623 & 893 & 427 & 12281 & 13 & 0 \\ 
  29 & 157 & 1805 & 1007 & 465 & 13587 & 13 & 0 \\ 
  30 & 148 & 1746 & 973 & 440 & 13049 & 15 & 0 \\ 
   \hline
\end{tabular}
\caption{A sideways table} 
\label{tabsmall}
\end{sidewaystable}\end{Schunk}
\clearpage

\subsection{Rescaled tables}
Specify a {\tt scalebox} value to rescale the table.

\begin{Schunk}
\begin{Sinput}
> x <- x[1:20,]
> x.rescale <- xtable(x,label='tabrescaled',caption='A rescaled table', digits = 0)
\end{Sinput}
\end{Schunk}

\begin{Schunk}
\begin{Sinput}
> print(x.rescale, scalebox=0.7)
\end{Sinput}
% latex table generated in R 3.1.1 by xtable 1.7-3 package
% 
\begin{table}[ht]
\centering
\scalebox{0.7}{
\begin{tabular}{rrrrrrrr}
  \hline
 & DriversKilled & drivers & front & rear & kms & VanKilled & law \\ 
  \hline
1 & 107 & 1687 & 867 & 269 & 9059 & 12 & 0 \\ 
  2 & 97 & 1508 & 825 & 265 & 7685 & 6 & 0 \\ 
  3 & 102 & 1507 & 806 & 319 & 9963 & 12 & 0 \\ 
  4 & 87 & 1385 & 814 & 407 & 10955 & 8 & 0 \\ 
  5 & 119 & 1632 & 991 & 454 & 11823 & 10 & 0 \\ 
  6 & 106 & 1511 & 945 & 427 & 12391 & 13 & 0 \\ 
  7 & 110 & 1559 & 1004 & 522 & 13460 & 11 & 0 \\ 
  8 & 106 & 1630 & 1091 & 536 & 14055 & 6 & 0 \\ 
  9 & 107 & 1579 & 958 & 405 & 12106 & 10 & 0 \\ 
  10 & 134 & 1653 & 850 & 437 & 11372 & 16 & 0 \\ 
  11 & 147 & 2152 & 1109 & 434 & 9834 & 13 & 0 \\ 
  12 & 180 & 2148 & 1113 & 437 & 9267 & 14 & 0 \\ 
  13 & 125 & 1752 & 925 & 316 & 9130 & 14 & 0 \\ 
  14 & 134 & 1765 & 903 & 311 & 8933 & 6 & 0 \\ 
  15 & 110 & 1717 & 1006 & 351 & 11000 & 8 & 0 \\ 
  16 & 102 & 1558 & 892 & 362 & 10733 & 11 & 0 \\ 
  17 & 103 & 1575 & 990 & 486 & 12912 & 7 & 0 \\ 
  18 & 111 & 1520 & 866 & 429 & 12926 & 13 & 0 \\ 
  19 & 120 & 1805 & 1095 & 551 & 13990 & 13 & 0 \\ 
  20 & 129 & 1800 & 1204 & 646 & 14926 & 11 & 0 \\ 
   \hline
\end{tabular}
}
\caption{A rescaled table} 
\label{tabrescaled}
\end{table}\end{Schunk}

\subsection{Table Width}
The {\tt tabularx} tabular environment provides more alignment options,
and has a {\tt width} argument to specify the table width.

Remember to insert \verb|\usepackage{tabularx}| in your \LaTeX preamble.

\begin{Schunk}
\begin{Sinput}
> df.width <- data.frame(
+   "label 1 with much more text than is needed" = c("item 1", "A"),
+   "label 2 is also very long" = c("item 2","B"),
+   "label 3" = c("item 3","C"),
+   "label 4" = c("item 4 but again with too much text","D"),
+   check.names = FALSE)
> x.width <- xtable(df.width,
+   caption="Using the 'tabularx' environment")
> align(x.width) <- "|l|X|X|l|X|"
\end{Sinput}
\end{Schunk}

\begin{Schunk}
\begin{Sinput}
> print(x.width, tabular.environment="tabularx",
+   width="\\textwidth")
\end{Sinput}
% latex table generated in R 3.1.1 by xtable 1.7-3 package
% 
\begin{table}[ht]
\centering
\begin{tabularx}{\textwidth}{|l|X|X|l|X|}
  \hline
 & label 1 with much more text than is needed & label 2 is also very long & label 3 & label 4 \\ 
  \hline
1 & item 1 & item 2 & item 3 & item 4 but again with too much text \\ 
  2 & A & B & C & D \\ 
   \hline
\end{tabularx}
\caption{Using the 'tabularx' environment} 
\end{table}\end{Schunk}

\section{Suppressing Printing}
By default the {\tt print} method will print the LaTeX or HTML to standard
output and also return the character strings invisibly.  The printing to
standard output can be suppressed by specifying {\tt print.results = FALSE}.

\begin{Schunk}
\begin{Sinput}
> x.out <- print(tli.table, print.results = FALSE)
\end{Sinput}
\end{Schunk}

Formatted output can also be captured without printing with the
{\tt toLatex} method.  This function returns an object of class
{\tt "Latex"}.

\begin{Schunk}
\begin{Sinput}
> x.ltx <- toLatex(tli.table)
> class(x.ltx)
\end{Sinput}
\begin{Soutput}
[1] "Latex"
\end{Soutput}
\begin{Sinput}
> x.ltx
\end{Sinput}
\begin{Soutput}
% latex table generated in R 3.1.1 by xtable 1.7-3 package
% 
\begin{table}[ht]
\centering
\begin{tabular}{|rr|lp{3cm}l|r|}
  \hline
 & grade & sex & disadvg & ethnicty & tlimth \\ 
  \hline
1 & 6 & M & YES & HISPANIC & 43 \\ 
  2 &  7 & M & NO & BLACK & 88 \\ 
  3 &   5 & F & YES & HISPANIC &  34 \\ 
  4 &    3 & M & YES & HISPANIC &   65 \\ 
  5 &     8 & M & YES & WHITE &    75 \\ 
  6 & 5 & M & NO & BLACK & 74 \\ 
  7 &  8 & F & YES & HISPANIC & 72 \\ 
  8 &   4 & M & YES & BLACK &  79 \\ 
  9 &    6 & M & NO & WHITE &   88 \\ 
  10 &     7 & M & YES & HISPANIC &    87 \\ 
   \hline
\end{tabular}
\end{table}
\end{Soutput}
\end{Schunk}

\section{Acknowledgements}
Most of the examples in this gallery are taken from the {\tt xtable} documentation.
\section{R Session information}
\begin{Schunk}
\begin{Sinput}
> #toLatex(sessionInfo())
\end{Sinput}
\end{Schunk}
\end{document}
